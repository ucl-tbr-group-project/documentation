\begin{frame}
	\frametitle{Conclusion}

	Decoupled approach:

	\begin{itemize}
		\item
			Tuned and compared surrogates from \alert{9~state-of-the-art
			families}.
		\item
			Found heuristic: \alert{GBTs} for $<10^4$~samples,
			\alert{ANNs} for $\geq10^5$~samples.
		\item
			Fastest found surrogate predicts TBR with standard deviation of
			error~$\num{0.033}$ in~\SI{0.898}{\micro\second}, which is~\alert{$8\cdot
			10^6\times$~faster} than Paramak.
		\item
			While this used 500K~samples, we found surrogates with
			comparable properties with \alert{as little as 10K~samples}.
	\end{itemize}

	\vspace{0.5em}

	Adaptive approach (on toy theory):
	\begin{itemize}
	    \item
	        New theoretical approach QASS developed, based on MCMC.
		\item 
			\alert{$60\%$ decrease} in evaluation MAE demonstrated.
		\item
			\alert{$6\%$ decrease} in expensive TBR samples needed.
		\item
			Strong potential for further reduction via hyperparameter tuning.
	\end{itemize}

	\vspace{0.5em}

	All presented methods \alert{portable} $\longrightarrow$ can be used as cheap
	approximation of any simulation or black box function.
\end{frame}


\documentclass[12pt]{iopart}

\usepackage{iopams}

\makeatletter
\@namedef{ver@amsmath.sty}{}
\makeatother

\usepackage{siunitx}
\usepackage{isotope}
\usepackage{fixltx2e}
\usepackage{booktabs}
\usepackage{xcolor}
\usepackage{multirow}
\usepackage{url}
\usepackage{graphicx}
\usepackage{tikz}
\usepackage{tabularx}
\usepackage[all=normal,bibbreaks=tight,floats=tight,mathspacing=tight,wordspacing=tight]{savetrees}

\providecommand{\keywords}[1]{\textbf{\textit{Index terms---}} #1}
\newcommand{\DDO}{D\textsubscript{2}O}
\definecolor{datacolor}{RGB}{72,36,117}

\graphicspath{{./fig_final_report/}{./fig/}}

\newcommand{\surrogateplot}[6]{
\begin{tikzpicture}
	\node[anchor=south west,inner sep=0] at (0,0){\includegraphics[width=\textwidth]{#1}};
	\draw[black,fill=white,line
		width=0.05mm,align=right,font=\fontsize{5.5}{6.5}\selectfont]
		(0.42,#6) rectangle
		(2.55,2.7) node[pos=.5] {
			\\[0.1ex]
			$\mathrm{MAE} = \num{#2}$\\
			$S = \num{#3}$\\
			$R^2 = \num{#4}$\\
			$R^2_{\text{adj.}} = \num{#5}$
		};
	\draw[black,fill=white,line
		width=0.05mm,align=left,font=\fontsize{5.5}{7.5}\selectfont]
		(2.02,1.3) rectangle
		(4.3,0.38) node[pos=.5] {
			{\color{red}
			\leavevmode\leaders\hrule height 0.7ex
			depth\dimexpr0.4pt-0.7ex\hskip3pt\kern0pt
			\hskip1pt
			\leavevmode\leaders\hrule height 0.7ex
			depth\dimexpr0.4pt-0.7ex\hskip3pt\kern0pt
			} Ideal model\\
			{\leavevmode\leaders\hrule height 0.7ex
			depth\dimexpr0.4pt-0.7ex\hskip3pt\kern0pt
			\hskip1pt
			\leavevmode\leaders\hrule height 0.7ex
			depth\dimexpr0.4pt-0.7ex\hskip3pt\kern0pt
			} Trained model\\
			\hskip3pt\raisebox{0.25ex}{\scalebox{0.75}{\color{datacolor}\textbullet}}\hskip4pt Data
		};
\end{tikzpicture}}

\begin{document}

%%%%%%%%%%%%%%%%%%%%%%%%%%%%%%%%%%%%%%%%%%%%%%%%%%%%%%%%%%%%%%%%%%%%%%%%%%%%%%%
% FRONT MATTER
%%%%%%%%%%%%%%%%%%%%%%%%%%%%%%%%%%%%%%%%%%%%%%%%%%%%%%%%%%%%%%%%%%%%%%%%%%%%%%%

\title[Fast Regression of the Tritium Breeding Ratio in Fusion Reactors]{Fast Regression of the
Tritium Breeding Ratio in Fusion Reactors}

\author{P~Mánek$^{1,2}$, G~Van Goffrier$^1$, V~Gopakumar$^3$, N~Nikolaou$^1$, J~Shimwell$^3$ and I~Waldmann$^1$}

\address{$^1$ Department of Physics and Astronomy, University College London, Gower Street, London WC1E~6BT, UK}
\address{$^2$ Institute of Experimental and Applied Physics, Czech Technical University, Husova 240/5, Prague 110~00, Czech Republic}
\address{$^3$ UK Atomic Energy Authority, Culham Science Centre, OX14~3DB Abingdon, UK}

\eads{\mailto{petr.manek.19@ucl.ac.uk}, \mailto{graham.vangoffrier.19@ucl.ac.uk}}

\begin{abstract}
	The tritium breeding ratio (TBR) is an essential quantity for the design of
	modern and next-generation D-T fueled nuclear fusion reactors. Representing the
	ratio between tritium fuel generated in breeding blankets and fuel consumed
	during reactor runtime, the TBR depends on reactor geometry and material
	properties in a complex manner. In this work, we explored the
	training of surrogate models to produce a cheap but high-quality approximation
	for a Monte Carlo TBR model in use at the UK Atomic Energy Authority. We
	investigated possibilities for dimensional reduction of its feature space, reviewed
	9~families of surrogate models for potential
	applicability, and performed hyperparameter optimisation. Here we present the
	performance and scaling properties of these
	models, the fastest of which, an artificial neural network,
	demonstrated~$R^2=\num{0.985}$ and a mean
	prediction time of~$\SI{0.898}{\micro\second}$, representing a relative speedup of $8\cdot 10^6$
	with respect to the expensive MC model. We further present a novel adaptive
	sampling algorithm, Quality-Adaptive Surrogate Sampling, capable
	of interfacing with any of the individually studied surrogates. Our preliminary
	testing on a toy TBR theory has demonstrated the efficacy of this algorithm for
	accelerating the surrogate modelling process.
\end{abstract}

\keywords{machine learning, surrogate model, regression, fast approximation,
Monte Carlo simulation, OpenMC, Paramak, tritium breeding, adaptive sampling}
\submitto{\NF}
\maketitle
\ioptwocol

%%%%%%%%%%%%%%%%%%%%%%%%%%%%%%%%%%%%%%%%%%%%%%%%%%%%%%%%%%%%%%%%%%%%%%%%%%%%%%%
% MAIN MATTER
%%%%%%%%%%%%%%%%%%%%%%%%%%%%%%%%%%%%%%%%%%%%%%%%%%%%%%%%%%%%%%%%%%%%%%%%%%%%%%%

\section{Introduction}
\label{sec:introduction}
\begin{frame}
	\frametitle{Project Background}
	Nuclear fusion -- the energy of the future!
    \vspace{10pt}
	\begin{itemize}
	    \item Must produce and contain an extremely hot and dense plasma
	    \begin{itemize}
		    \item Magnetic Confinement Fusion (MCF): toroidal circulation
		    \item Inertial Confinement Fusion (ICF): spherical compression
		\end{itemize}
		\vspace{10pt}
		\item Modern designs require enriched Hydrogen fuel of two varieties:
	    \begin{itemize}
		    \item Deuterium ($^2$H) -- abundant in naturally-sourced water
		    \item Tritium ($^3$H) -- extremely rare, but can be produced \textit{in-reactor}
		\end{itemize}
	\end{itemize}
	\vspace{10pt}
	\centering{\includegraphics[height=3cm]{icf_diagram}}
\end{frame}

\begin{frame}
	\frametitle{Problem Description}
	\begin{itemize}
		\item % TODO
	\end{itemize}
\end{frame}

\begin{frame}
	\frametitle{Data Generation}
	\begin{itemize}
		\item % TODO
	\end{itemize}
\end{frame}

%\begin{frame}
%	\frametitle{Dimensionality Reduction}
%	\begin{itemize}
%		\item % TODO
%	\end{itemize}
%\end{frame}

\begin{frame}
	\frametitle{Methodology}
			Conventional regression task -- search for a cheap surrogate $\hat{f}(x)$ that
			minimizes dissimilarity with an expensive function $f(x)$:

			\begin{itemize}
				\item
					Regression performance (capability to approximate)
					\begin{itemize}
						\item Absolute: mean absolute error, $\sigma$ of error
						\item Relative: $R^2$, $R^2_\text{adj.}$
					\end{itemize}
				\item
					Computational complexity:
					wall training \& prediction time / sample.
			\end{itemize}

			2 approaches for surrogate training:
			\begin{enumerate}
				\item
					Decoupled -- trains models from previously sampled
					$\mathcal{T}=\{(x,f(x))\}$.
				\item
					Adaptive -- repeats sampling \& model training, increases
					sampling density in low-performance regions.
			\end{enumerate}
\end{frame}



\section{Methodology}
\label{sec:methodology}
Assuming that input has been appropriately treated to eliminate redundant
features, we may turn to characterise proposed surrogate models and the criteria
used for their evaluation. The task all presented surrogates strive to solve can be
formulated using the language of conventional regression problems. In the scope
of this work, we explore various possible choices available to us in the
scheme of supervised and unsupervised learning.

Labeling the expensive Monte Carlo simulation $f(x)$, a surrogate is a mapping
$\hat{f}(x)$ that yields similar images as $f(x)$. In other words, $f(x)$ and
$\hat{f}(x)$ minimise a selected similarity metric. Furthermore, in order to
be considered \textit{viable}, surrogates are required to achieve expected evaluation time
that does not exceed the expected evaluation time of $f(x)$.

In the supervised learning setting, we first gather a sufficiently large
training set of samples $\mathcal{T}=\left\{\left( x^{(i)},f\left(x^{(i)}\right) \right)\right\}_{i=1}^N$
to describe the behaviour of $f(x)$ across its domain.
Depending on specific model class and appropriate choice of its
hyperparameters, surrogate models $\hat{f}(x)$ are trained to minimise
empirical risk with respect to $\mathcal{T}$ and a model-specific
loss function $\mathcal{L}$, where empirical risk is defined as

\begin{align}
	R_{\text{emp.}}(\hat{f}\mid\mathcal{T},\mathcal{L})
	=\frac{1}{N}\sum_{i=1}^N
	\mathcal{L}\left(\hat{f}(x^{(i)}),f(x^{(i)})\right).
\end{align}

The unsupervised setting can be viewed as an extension of this method.
Rather than fixing the training set $\mathcal{T}$ for the entire duration of
training, multiple sets $\{\mathcal{T}_k\}_{k=0}^K$ are used, such that
$\mathcal{T}_{k-1}\subset\mathcal{T}_k$ for all $k>1$. The first set
$\mathcal{T}_0$ is initialised randomly to provide a \textit{burn-in}, and is
repeatedly extended in epochs, whereby each epoch trains a new surrogate on
$\mathcal{T}_k$ using the supervised learning procedure, evaluates its
performance, and forms a new set $\mathcal{T}_{k+1}$ by adding more samples to
$\mathcal{T}_k$. This permits the learning algorithm to condition the selection
of new samples by the results of evaluation in order to focus on improvement of
surrogate performance in complex regions within the domain.


\subsection{Metrics}
\label{sec:metrics}

Aiming to provide objective comparison of a diverse set of surrogate model
classes, we define a multitude of metrics to be tracked during experiments.
Following the motivation of this work, two desirable properties of surrogates
arise: (i) their capability to approximate the expensive
model well and (ii) their time of evaluation. An ideal surrogate would maximise
the former while minimising the latter.

\Cref{tbl:metrics} provides exhaustive list and description of metrics recorded
in the experiments. For regression performance analysis, we include a selection
of absolute metrics to assess the approximation capability of surrogates, and set
practical bounds on the expected accuracy of their predictions. In addition, we also track
relative measures that are better-suited for model comparison between works as
they maintain invariance with respect to the domain and image space.

\begin{table}[h]
	\centering
	\begin{tabular}{lll}
	\toprule
	Regression performance metrics	& Mathematical formulation / description & Ideal value \\
	\midrule
	Mean absolute error (MAE)	& $\sum_{i=1}^N |y^{(i)}-\hat{y}^{(i)}|/N$ & 0
	[TBR] \\
	Standard error of regression $S$	& $\text{StdDev}_{i=1}^N\left( |y^{(i)} -
	\hat{y}^{(i)}| \right) $	 & 0 [TBR] \\
	$R^2$ ratio (coefficient of determination)	& $1-\sum_{i=1}^N \left(y^{(i)}-\hat{y}^{(i)} \right)^2 /
	\sum_{i=1}^N \left( y^{(i)}-\overline{y} \right)^2 $ & 1 [rel.] \\
	Adjusted $R^2$ ratio	& $1-(1-R^2)(N-1)/(N-P-1)$	& 1 [rel.] \\
	\midrule
	Evaluation time metrics	& {} & {} \\
	\midrule
	Mean sample training time	& $(\text{wall training time of
	$\hat{f}(x)$})/N_0$ 	& 0 [ms] \\
	Mean sample prediction time	& $(\text{wall evaluation time of
	$\hat{f}(x)$})/N$	& 0 [ms] \\
	\bottomrule
	\end{tabular}
	\caption{Metrics recorded in supervised learning experiments. In
	formulations, we work with training set of size $N_0$ and testing set of
size $N$, TBR values $y^{(i)}=f(x^{(i)})$ and $\hat{y}^{(i)}=\hat{f}(x^{(i)})$
denote images of the $i$th testing sample in the expensive model and the surrogate
respectively. Furthermore, the mean $\overline{y}=\sum_{i=1}^N y^{(i)}/N$ and $P$ is the
number of input features.}
	\label{tbl:metrics}
\end{table}

To prevent undesirable bias in results due to training set selection, all metrics
collected in the scheme of $k$-fold cross-validation with a standard choice of
$k=5$. Herein, a sample set is subdivided into 5 disjoint folds which are
repeatedly interpreted as training and testing sets, maintaining a constant
ratio of samples between the two. In each such interpretation experiments are
repeated, and the overall value of each metric of interest is reported as the
mean across all folds.


\subsection{Model Comparisons}
\label{sec:model}


\subsection{Adaptive Sampling}
\label{sec:adaptive}



\section{Results}
\label{sec:results}
\subsection{Results of Model Comparisons}
\label{sec:modelres}



\newpage

\subsection{Results of Adaptive Sampling}
\label{sec:adaptiveres}

Define sinusoidal toy model and justify

Explain hyperparameter tests: initsamples, stepsamples, MCMC length

\begin{figure}[h]
    \centering
    \begin{subfigure}[t]{0.5\textwidth}
        \centering
        \includegraphics[width=1.1\linewidth]{fig5_qassincrsamp.png}
    \end{subfigure}%
    ~ 
    \begin{subfigure}[t]{0.5\textwidth}
        \centering
        \includegraphics[width=1.1\linewidth]{fig6_qassincrtime.png}
    \end{subfigure}
    \caption{QASS absolute training error over total sample quantity (left) and number of iterations (right)}
\end{figure}

\begin{figure}[h]
  \centering
    \includegraphics[width=0.8\linewidth]{fig7_qasssampling.png}
    \caption{Absolute training error for QASS, uniform random scheme, and mixed scheme}
  \label{fig:pca}
\end{figure}


\section{Conclusion}
\label{sec:conclusion}
\begin{frame}
	\frametitle{Conclusion}

	Decoupled approach:

	\begin{itemize}
		\item
			Tuned and compared surrogates from 9~state-of-the-art families.
		\item
			Found heuristic: GBTs for $<10^4$~points and
			ANNs for $\geq10^5$~points.
		\item
			Fastest found surrogate predicts TBR with standard deviation of
			error~$\num{0.033}$ in~\SI{0.898}{\micro\second}, which is~$8\cdot
			10^6\times$~faster than Paramak.
		\item
			While this used 500K~datapoints, we found surrogates with
			comparable properties with as little as 10K~datapoints.
	\end{itemize}

	\vspace{0.5em}

	Adaptive approach:
	\begin{itemize}
		\item 
			TODO % TODO Graham
		\item
			TODO % TODO Graham
		\item
			TODO % TODO Graham
		\item
			TODO % TODO Graham
	\end{itemize}

	\vspace{0.5em}

	Presented methods portable $\rightarrow$ can be used as cheap
	approximation of any simulation or black box function.
\end{frame}




%%%%%%%%%%%%%%%%%%%%%%%%%%%%%%%%%%%%%%%%%%%%%%%%%%%%%%%%%%%%%%%%%%%%%%%%%%%%%%%
% BACK MATTER
%%%%%%%%%%%%%%%%%%%%%%%%%%%%%%%%%%%%%%%%%%%%%%%%%%%%%%%%%%%%%%%%%%%%%%%%%%%%%%%

\section{Acknowledgements}
\label{sec:acknowledgements}
PM was supported the STFC UCL Centre for Doctoral Training in Data Intensive
Science (grant no. ST/P006736/1).


\section{References}
\label{sec:references}
\bibliography{ucl_tbr_paper}
\bibliographystyle{iopart-num}


\end{document}


\documentclass[12pt]{article}
\usepackage{amsfonts}
\usepackage{amsmath,amssymb}
\usepackage{color}
\usepackage{graphicx}

\topmargin = 1mm
\headsep = 1mm
\textheight = 228mm
\textwidth = 167.0mm
\oddsidemargin = 1.1mm
\evensidemargin = 1.1mm
\date{\today}

% Set the format for notes on in-progress responses
\newcommand\myNotes[1]{\textcolor{red}{#1}}
\newcommand\myTodo[1]{\textcolor{blue}{#1}}

\begin{document}

\begin{center}
\bf Reviewer Response for Manuscript MLST-100670
\end{center}

\bigskip

\noindent
Dear Editor,

\bigskip
\noindent
We thank the referee for his or her careful consideration and useful comments.
Please find below our detailed responses. We have also made the corresponding
changes in the text.

\bigskip
\noindent
Best Regards,\\
Petr Mánek (for the authors)

\bigskip
\bigskip
\bigskip
\noindent
{\bf Comments of Referee 1}

\begin{enumerate}

\item Page 3, Section 1.1, Table 1: Please correct abbreviation \textit{Blanket
	Structural Fraction (BSM)} defined in Table 1. It should be BSF. While BSM
	corresponds to \textit{``Blanket Structural Material''}.

{\bf
	The ``BSM'' abbreviation was indeed intended to denote ``Blanket
	Structural Material'', as can be seen further on in Table 4, where it is
	associated with ``eurofer''. In the submitted manuscript, the abbreviation
	was incorrectly misplaced in Table 1 by one row. We have corrected this
	error.
}

\item Page 3, Section 1.1, lines 47-54: Please clarify the meaning of the
	sentence which currently is trivial and does not convey much sense:
	\textit{``This represents a significant step forwards in computational
		fusion-reactor design, as any speed-up achieved in TBR evaluation
		directly informs a speed-up in numerical optimization of reactor
		parameters, although such optimization is beyond the scope of the
		present work.''} Particularly, I suspect that its second clause is very
		doubtful: \textit{``as any speed-up achieved in TBR evaluation directly
		informs a speed-up in numerical optimization of reactor parameters,''}
		Maybe, the problems in understanding are related with the phrases
		\textit{``any speed-up''} -- this is too general expression, the second
		phrase is \textit{``directly informs''} -- maybe you wanted to say
		\textit{``directly connected''}?

{\bf
	It is true that the wording of this sentence is perhaps unnecessarily
	complex. Its purpose is to motivate our work in the following sense: if a
	computationally expensive Monte Carlo (MC) simulation is required in order
	to evaluate a single instance of reactor parameters (i.e.~a single data
	point of some phase space), it may act as a performance bottleneck
	prohibiting efficient exploration of that domain. Our surrogate models can
	be viewed as a possible improvement of this paradigm. If a surrogate model
	is trained to be sufficiently accurate, it could perhaps completely
	substitute MC simulations in this process (note that MC would still be
	needed to train the model). Or alternatively, one could envision a hybrid
	optimization process where a surrogate model acts as a heuristic that guides
	the optimization algorithm, prunes the parameter space and reduces the
	number of calls to the MC. For instance, such approach would be consistent
	with Branch \& Bound strategy. Since the prediction time of the presented
	surrogate models is significantly smaller than the runtime of the MC, in
	both cases we would observe a relative speed-up of the optimization process.

	We have rephrased the referenced sentence to better convey this meaning.
}

\end{enumerate}

\end{document}

The analysis of massive datasets has become a necessary component of virtually all technical fields, as well as the social and humanistic sciences, in recent years. Given that rapid improvements in sensing and processing hardware have gone hand in hand with the data explosion, it is unsurprising that software for the generation and interpretation of this data has also attained a new frontier in complexity. In particular, simulation procedures such as Monte Carlo (MC) event generation can perform physics predictions even for theoretical regimes which are not analytically soluble. The bottleneck for such procedures, as is often the case, lies in the computational time and power which they necessitate.

Surrogate models, or metamodels, can resolve this limitation by replacing a resource-expensive procedure with a much cheaper approximation [1]. They are especially useful in applications where numerous evaluations of an expensive procedure are required over the same or similar domains, e.g. in the parameter optimisation of a theoretical model. The term "metamodel" proves especially meaningful in this case, when the surrogate model approximates a computational process which is itself a model for a (perhaps unknown) physical process [2]. There exists a spectrum between "physical" surrogates which are constructed with some contextual knowledge in hand, and "empirical" surrogates which are derived purely from the underlying expensive model. 

In this internship project, in coordination with the UK Atomic Energy Authority (UKAEA) and Culham Centre for Fusion Energy (CCFE), we sought to develop a surrogate model for the tritium breeding ratio (TBR) in a Tokamak-class nuclear fusion reactor. Our expensive model was a MC-based neutronics simulation [3], itself a spherical approximation of the Joint European Torus (JET) at CCFE, which returns a prediction of the TBR for a given reactor configuration. We took an empirical approach to the construction of this surrogate, and no results described here are explicitly dependent on prior physics knowledge.

For the remainder of Section 1, we will define the TBR and set the context of this work within the goals of the UKAEA and CCFE. In Section 2 we will describe our datasets generated from the expensive model for training and validation purposes, and the dimensionality reduction methods employed to develop our understanding of the parameter domain. In Section 3 we will present our methodologies for the comparison testing of a wide variety of surrogate modelling techniques, as well as a novel adaptive sampling procedure suited to this application. After delivering the results of these approaches in Section 4, we will give our final conclusions and recommendations for further work.

\subsection{Problem Description}
\label{sec:problemdescription}

- Introduce nuclear fusion via tokamak [4][5]

- details of JET / CCFE [6]

-------------------------------------------------------------------------------------

- Describe tritium breeding [4]
    - Modelling of reactor geometries [7]
    















[1] Sondergaard 2003
[2] Myers and Montgomery 2002
[3] Collaboration with Jonathan 2020

[4] Hernandez 2018
[5] Tokamak wikipedia 
[6] Keilhacker 1999
[7] Coleman 2019

